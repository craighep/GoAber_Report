\chapter{Introduction}
This report looks into my personal views of the group project completed for the Go!Aber application. I will be evaluating the group performance, as well as my own personal performance in the following two chapters. I will provide critique and possible improvements for various times in the project, outlining what I felt when well, and what could have gone better.  Throughout this evaluation, I will be using my personal blog that I wrote each week. The blog entries can be found in appendix \par
APPENFIX REFERENCE HERE FOR BLOG
\section{Personal evaluation}
To start by looking at my own contribution to the project, this began during the initial reading of the requirements specification. Alongside the other group members, I carefully read through and provided description about each specification, noting down any possible questions or issues. Following the minutes recorded (this was done by myself throughout the project), questions recorded were then put to the client. Again, I was responsible for creating the minutes from that meeting, and were placed in the project folder for access by any other members, client or project manager.\par
Staying on the subject of my role during meetings, in each sprint review with the project manager, I ensured that both the meeting room was free and also that major statistics such as burndown, sprint tasks completed, and hours worked were explained. This was an important task for me, as I was selected as scrum master. I was the member to suggest using scrum as the principle development methodology, following from my personal experiences during my industrial year. \par
I felt that both scrum, and my role as scrum master were both handled well, and that I ensured at the beginning and end of each sprint that the team as a whole were on track to complete the weeks’ tasks. Perhaps one improvement that could have bettered my performance as scrum master could have been to explain the roles and rules of the methodology better to other members after the workflow was selected. This could have improved the use of TFS (Microsoft visual studio online), where instead some members were slightly late on creating new tasks, and marking others completed when required.\par
Furthering the scrum responsibilities I held, I am happy with the way that I managed meetings for sprint planning. Where meetings had time restraints, I ensured that the team did not discuss certain issues for too long (unless necessary). For instance, in initial discussions of the class diagram, I suggested that because we were still only at the stage of designing the database, that the team should not get too far ahead, and take more time concentrating on current work. \par
Finally for my scrum master role, I worked closely with all other team members to ensure that the most suitable tasks were placed with a higher priority, therefore ensuring these tasks were completed first. For instance, creating a stable database was the first task that I agreed had to be completed before any other task could be started.\par
Another couple of tasks that I knew had to be completed early on in the project was the development of the Fitbit and Jawbone connections. Because I had some experience with API connections and SOAP/ Rest connectivity, I put myself forward for handling these tasks. Starting by browsing a few examples on the internet and some that were suggested in the practical lectures of this module, I was able to create external projects capable of connecting to the Fitbit servers from .NET. This was completed within the second week of the project, and the first sprint week.
